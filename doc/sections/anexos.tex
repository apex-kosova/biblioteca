\section{Anexos}
\subsection{Repositorio en GitHub}
El desarrollo del proyecto se encuentra en la siguiente dirección: \url{https://github.com/jrsnzc/biblioteca}.

\subsection{Instalación y ejecución de Oracle APEX}
Iniciamos la máquina virtual con el comando \lstinline$vagrant up$.
\begin{figure}[H]
  \centering
  \includegraphics[width=0.9\textwidth]{Vagrant-APEX}
  \caption{Levantamiento de procesos de la máquina virtual}
  \label{fig:vagrant-apex}
\end{figure}

Creamos una nueva estación de trabajo y un usuario administrador.
\begin{figure}[H]
  \centering
  \includegraphics[width=0.9\textwidth]{APEX-login}
  \caption{Oracle APEX, pantalla de inicio de sesión}
  \label{fig:apex-login}
\end{figure}

\subsection{Programa: Sistema de gestión de biblioteca}
\subsubsection{Formularios}
En la figura \ref{fig:biblio-form_libro} se muestra los datos a llenar al crear un nuevo libro. Se elije a que editorial y categoría pertenece.
\begin{figure}[H]
  \centering
  \includegraphics[width=0.95\textwidth]{biblio-form_libro}
  \caption{Formulario para la creación de libros}
  \label{fig:biblio-form_libro}
\end{figure}
En la figura \ref{fig:biblio-form_editorial} se muestra los datos a llenar para crear una nueva editorial.
\begin{figure}[H]
  \centering
  \includegraphics[width=0.95\textwidth]{biblio-form_editorial}
  \caption{Formulario para la creación de una nueva editorial}
  \label{fig:biblio-form_editorial}
\end{figure}
En la figura \ref{fig:biblio-form_categoria} se muestra los datos a llenar para crear una nueva categoría.
\begin{figure}[H]
  \centering
  \includegraphics[width=0.95\textwidth]{biblio-form_categoria}
  \caption{Formulario para la creación de una nueva categoría}
  \label{fig:biblio-form_categoria}
\end{figure}
En la figura \ref{fig:biblio-form_autor} se muestra los datos para agregar un nuevo autor al sistema de biblioteca.
\begin{figure}[H]
  \centering
  \includegraphics[width=0.95\textwidth]{biblio-form_autor}
  \caption{Formulario para la añadir un nuevo autor}
  \label{fig:biblio-form_autor}
\end{figure}
En la figura \ref{fig:biblio-form_alumno} se muestra los datos para registrar un nuevo alumno.
\begin{figure}[H]
  \centering
  \includegraphics[width=0.95\textwidth]{biblio-form_alumno}
  \caption{Formulario para la creación de alumnos}
  \label{fig:biblio-form_alumno}
\end{figure}
En la figura \ref{fig:biblio-form_prestamo} se muestra los datos que se necesitan al prestar un libro a un alumno. Se elije de una lista de ejemplares disponibles y CUI de los estudiantes.
\begin{figure}[H]
  \centering
  \includegraphics[width=0.95\textwidth]{biblio-form_prestamo}
  \caption{Formulario para la asignación de un ejemplar a un alumno}
  \label{fig:biblio-form_prestamo}
\end{figure}
\subsubsection{Reportes}
En la figura \ref{fig:biblio-report_libro} se muestra los libros con los que cuenta la biblioteca.
\begin{figure}[H]
  \centering
  \includegraphics[width=0.95\textwidth]{biblio-report_libro}
  \caption{Reporte de libros}
  \label{fig:biblio-report_libro}
\end{figure}
En la figura \ref{fig:biblio-report_alumno} se muestra los estudiantes registrados en la biblioteca.
\begin{figure}[H]
  \centering
  \includegraphics[width=0.95\textwidth]{biblio-report_alumno}
  \caption{Reporte de alumnos registrados}
  \label{fig:biblio-report_alumno}
\end{figure}
\subsubsection{Vistas}
En la figura \ref{fig:biblio-report_prestamo} se muestra los ejemplares de la biblioteca que están en estado de préstamo.
\begin{figure}[H]
  \centering
  \includegraphics[width=0.95\textwidth]{biblio-report_prestamo}
  \caption{Reporte de ejemplares prestados}
  \label{fig:biblio-report_prestamo}
\end{figure}